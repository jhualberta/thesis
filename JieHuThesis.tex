\documentclass[phd,black]{PrincetonThesis}
\usepackage{epsfig}
\usepackage{epigraph}
\usepackage{hyperref}
\usepackage{etoolbox}
\usepackage{amsmath,amssymb}
\usepackage{isotope}
\usepackage{booktabs}
\usepackage{longtable}
\usepackage{subfigure}
\usepackage{algorithm}
\usepackage{algorithmic}
%%%%for flow charts
%%\usepackage{tikz}
%%\usetikzlibrary{shapes.geometric, arrows}

\usepackage{multirow}

%\usepackage[frenchb]{babel}
\makeatletter
\newlength\epitextskip
\pretocmd{\@epitext}{\em}{}{}
\apptocmd{\@epitext}{\em}{}{}
\patchcmd{\epigraph}{\@epitext{#1}\\}{\@epitext{#1}\\[\epitextskip]}{}{}
\makeatother

\title{A Measurement of Solar Neutrinos and the Development of Reconstruction Algorithms for the SNO+ Experiment}
\author{Jie Hu}
\department{Physics}
\advisor{Aksel Hallin}
\degreemonth{December}
\degreeyear{2021}

%\usechemmodule{isotopes} 
\begin{document}

\begin{frontmatter}
  
\begin{thesisabstract}
\input abstract.tex
\end{thesisabstract}

%\begin{preface}
%	\input preface.tex
%\end{preface}

\begin{acknowledgements}
\input acknowledgements.tex
\end{acknowledgements}
 
\end{frontmatter}

\cleardoublepage
\chapter{Introduction}
\input chapter1intro.tex
\chapter{Neutrino physics}
\input chapter2nuPhys.tex
\chapter{The SNO+ Experiment}
\input chapter3detector.tex
\chapter{Event Reconstruction}
\input chapter4recon.tex
\chapter{Calibration}
\input chapter5calibration.tex
\chapter{Solar Neutrino Analysis in the SNO+ Water Phase}
\input chapter6waterAnaly.tex
\chapter{Conclusions}
\input conclusion.tex

\appendix
\cleardoublepage
\chapter{Details for the MultiPath Fitter}
\input appendix.tex
%\chapter{Information for $^{16}$N Scan Runs}
\input appendix16N.tex
\chapter{Bi-Po Analysis}
\input appendixBiPo.tex
%\chapter{Run Lists Used for Water Analysis}
%\input appendixWaterRuns.tex

\cleardoublepage
%\nocite{*} %% use everything in bibtex file, even if not cited
%\bibliographystyle{abbrv} %% or your favorite style
\bibliographystyle{ieeetr}
\bibliography{thesis} %% assuming your bibtex file is thesis.bib	
\end{document}