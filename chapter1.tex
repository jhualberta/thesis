%\epigraph{Yesterday's rose stands only in name, we hold only empty names.}{--- \textup{Umberto Eco}, \textit{The Name of Rose}}
 

 Neutrinos are one of the elementary particles we currently know and are included in the Standard Model (SM). However, some properties of neutrinos can not be described by the SM, which shows clues of the new physics beyond the Standard Model.

SNO+ experiment is planned to explore one of the unknown properties of neutrinos: whether the neutrinos are Majorana particles or Dirac particles.



Massive neutrinos are discussed ...
\section{Neutrino Oscillation}

Neutrino oscillation was first discovered in 1998.
It is the first direct evidence showing that the Standard Model is incomplete. 


solar neutrino oscillations




matter effect





\section{Majorana Neutrino}

Dirac equation $(i\gamma^\mu\partial_\mu-m)\psi=0$,
get coupled equations


The interpretation of the $0\nu\beta\beta$ process is considered as exchanging light Majorana neutrinos. In this case the effective Majorana mass $<m_{ee}>=\sum_{i=1}^{3} |U_{ei}|^2m_i~(i=1,2,3)$, $U_{ei}$ are the elements of the neutrino mixing matrix for the flavor state $\nu_e$, and $m_i$ are the mass eigenvalues of the mass eigenstates (from (\ref{eq:mixingmatrix})). The observable quantity is the half-life:
\[
(T^{0\nu\beta\beta}_{1/2})^{-1} = G_{PS}(Q,Z)|M_{Nuclear}|^2<m_{ee}>^2, 
\]

Majorana found a representation of the $\gamma$-matrices as follow:
\[
\gamma_M^0 = \begin{pmatrix} 
0 & \sigma^2 \\
\sigma^2 & 0
\end{pmatrix},
\gamma_M^1 = \begin{pmatrix} 
\sigma^3 & 0 \\
0 & \sigma^3
\end{pmatrix},
\gamma_M^2 = \begin{pmatrix} 
-\sigma^2 & 0 \\
0 & \sigma^2
\end{pmatrix},
\gamma_M^3 = -i\begin{pmatrix} 
\sigma^1 & 0 \\
0 & \sigma^1
\end{pmatrix}
\]


These matrices themselves are pure imaginary. 




















The GERmanium Detector Array (GERDA) experiment searches for $0\nu\beta\beta$ of $^{76}$Ge. The experiment uses bare germanium crystals with an enrichment of up to $\sim$87\% $^{76}$Ge operated in a radiopure cryogenic liquid argon (LAr). GERDA Phase I had an exposure of 21.6 kg$\cdot$yr and Phase-II started with 35.6kg from enriched material in December 2015. With combined data of Phase I and Phase II, GERDA reported in 2016 a lower limit half-life of $T^{0\nu}_{1/2}(^{76}$Ge$)>5.3\times 10^{25}$ years at 90\% C.L.\cite{gerda,gerda2}.

The Enriched Xenon Observatory (EXO) experiment uses 200-kg liquid Xenon (LXe) time projection chamber (TPC) to search for $0\nu\beta\beta$ in $^{136}$Xe. In 2011 they observed the half life of double beta decay of $^{136}$Xe to be $2.11\times 10^{21}$ years and in 2014 they set a limit on $T^{0\nu}_{1/2}(^{136}$Xe$)>1.1\times 10^{25}$ yr\cite{exo}. EXO is now upgrading to the next 5-tonne experiment (nEXO) and is expected to reach an exclusion sensitivity of $T^{0\nu}_{1/2}(^{136}$Xe) to about $10^{28}$ years at 90\% C.L.\cite{nEXO}.

Also looking into $^{136}$Xe, the KamLAND-Zen experiment exploits the existing facilities of KamLAND by setting a 3.08-m-diameter spherical inner balloon filled with 13 tons of Xe-loaded liquid scintillator at the center of the KamLAND detector. Their 2016 results from a 504 kg$\cdot$yr exposure obtained a lower limit for the $0\nu\beta\beta$ decay half-life of $T^{0\nu}_{1/2}(^{136}$Xe$)>1.07\times 10^{26}$ yr at 90\% C.L. and the corresponding upper limits on the effective Majorana neutrino mass are in the range 61-165 meV\cite{kamlandZen}.

The Cryogenic Underground Observatory for Rare Events (CUORE) experiment searches for $0\nu\beta\beta$ in $^{130}$Te. CUORE is a ton-scale cryogenic bolometer array that arranges 988 tellurium dioxide (TeO$_2$) crystals. CUORE reported first results in 2017 after a total TeO$_2$ exposure of 86.3 kg$\cdot$yr. Combined with their early data, they placed a lower limit of $T^{0\nu}_{1/2}(^{130}$Te$)>1.5\times 10^{25}$ yr at 90\% C.L. and $m_{\beta\beta}<(140-400)$  meV\cite{cuore}.

