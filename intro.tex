%\epigraph{Yesterday's rose stands only in name, we hold only empty names.}{--- \textup{Umberto Eco}, \textit{The Name of Rose}}
 
A neutrino is a fermion with neutral electric charge and interacts only via weak interaction and gravity, which is described by the Standard Model. The Standard Model is a theory describing all of the elementary particles to our current knowledge and their interactions.
In the Standard Model, neutrinos are created from weak interactions in one of three leptonic flavors:
electron neutrinos ($\nu_e$), muon neutrinos ($\nu_\mu$) and tauon neutrinos ($\nu_\tau$), accompanied by electrons ($e$), muons ($\mu$) and tauons ($\tau$) respectively. The weak interactions are described by fermions exchanging $W^{\pm}$ and $Z^0$ bosons as weak force carriers.

The Standard Model has successfully explained and predicted phenomena in particle physics since the latter half of the 20th century, including the discovery of Higgs bosons in 2012, which is a crucial piece in the Standard Model. However, it has issues such as the requirement of input parameters which can not be determined by the theory itself. Moreover, there are a few questions and problems the Standard Model can not answer or solve. The mystery properties and behaviors of neutrinos contribute a few of those questions: What are the masses of neutrinos? How do neutrinos obtain their masses? Why their masses are so small compared to the other elementary particles? Are neutrinos their own antiparticles? And so on. To answer these questions about neutrinos will open a door to the new physics theories beyond the Standard Model.

Since neutrinos weakly interact with other particles and fields, they can penetrate through massive matter or travel a long way through the space without being interrupted. Neutrinos produced in the core of the Sun, in Supernovae, or in the galactic core of the Milky Way can carry original information of these astrophysics objects and easily reach the detectors on the Earth. This enables neutrinos as a probe to study the status of astrophysics objects.

These interesting facts put the researches of neutrinos under the spotlight. Among neutrino experiments, SNO+ is a multi-purpose experiment with a main goal to search for an extremely rare process called neutrinoless double beta decay. This will explore the unknown nature of neutrinos: whether they are Majorana or Dirac
particles, unravel the masses of neutrinos, and test the new physics theories as well.

SNO+ has measured the high energy components of the neutrinos from the Sun, or solar neutrinos along with an extremely low backgrounds. It will measure the low energy component of the solar neutrinos.

In this thesis, a set of position and time reconstruction algorithms is developed for multiple SNO+ physics phases. The algorithms have been tested in the calibration runs. They are applied in solar neutrino and backgrounds analysis.
