%\setlength{\epigraphwidth}{0.5\textwidth}
%\epigraph{``Even so," said Trevize, ``I must search. Even if the endless powdering of stars in the Galaxy makes the quest seem hopeless, and even if I must do it alone."}{--- \textup{Isaac Asimov}, \textit{Foundation and Earth }}

A neutrino is a fermion with neutral electric charge and interacts only via weak interaction and gravity, which is described by the Standard Model. The Standard Model is a theory describing all of the elementary particles to our current knowledge and their interactions.
In the Standard Model, neutrinos are created from weak interactions in one of three leptonic flavors:
electron neutrinos ($\nu_e$), muon neutrinos ($\nu_\mu$) and tauon neutrinos ($\nu_\tau$), accompanied by electrons ($e$), muons ($\mu$) and tauons ($\tau$) respectively. The weak interactions are described by fermions exchanging $W^{\pm}$ and $Z^0$ bosons as weak force carriers.

The Standard Model has successfully explained and predicted phenomena in particle physics since the latter half of the 20th century, including the discovery of Higgs bosons in 2012, which is a crucial piece predicted by the Standard Model. However, there are opening issues in the Standard Model, such as the requirement of input parameters which can not be determined by the theory itself. Moreover, there are a few questions and problems the Standard Model can not answer or solve. The mystery properties and behaviors of neutrinos contribute a few of those questions: What are the masses of neutrinos? How do neutrinos obtain their masses? Why their masses are so small compared to the other elementary particles? Are neutrinos their own antiparticles? And so on. To answer these questions about neutrinos will open a door to the new physics theories beyond the Standard Model.

Since neutrinos weakly interact with other particles and fields, they can penetrate through massive matter or travel a long way through the space without being interrupted. Neutrinos produced in the core of the Sun, in Supernovae, or in the galactic core of the Milky Way can carry original information of these astrophysics objects and easily reach the detectors on the Earth. This enables neutrinos as a probe to study the status of astrophysics objects.

These interesting facts put the researches of neutrinos under the spotlight. Among state-of-the-art neutrino experiments, SNO+ aims to search for an extremely rare process called neutrinoless double beta decay ($0\nu\beta\beta$). This search will explore the unknown nature of neutrinos: whether they are Majorana or Dirac particles, unravel the masses of neutrinos, and test the new physics theories as well.

The SNO+ experiment will go through three major stages, which are mainly determined by the working medium inside the SNO+ detector. First, the detector was filled with water and worked as a water Cherenkov detector. This period of data collection for physics research is called the ``water phase''. During this stage, certain fluxes of neutrinos from the Sun has been measured. Rare decay process modes, the invisible nucleon decay, which is predicted by the Grand Unified Theory (GUT) have also been searched. During this phase, calibration sources have been deployed for measuring the detector properties. Up until this thesis writing, the water phase stage has been completed and the water is gradually replaced by the liquid scintillator. Once the detector is fully filled with the liquid scintillator, the experiment will walk into the ``scintillator phase''.


After that, the Tellurium will be loaded into the liquid scintillator 


This transition stage is called ``partial-fill phase''.



Then I further developed and completed it for multiple SNO+ physics phases, especially for the ``partial-fill phase'', which refers to the transition statuses when the SNO+ detector is partially filled with the liquid scintillator.


In this thesis, a framework of event position and time reconstruction algorithms is developed for multiple SNO+ physics phases. This event reconstruction framework was first developed by the author's supervisor, Dr. Aksel Hallin, to reconstruct and investigate the data taken at an early stage of SNO+ when the detector was partially filled with water at the end of 2014 (called ``partial water fill stage'')\cite{partialWater}. It was further developed by the researchers from the University of Alberta: Dr. Kalpana Singh and Dr. David Auty, for investigating the wavelength shifter and water events\cite{davidPartialWater, kalpanaWLS, kalpanaWLS2, kalpanaMPFitter}. Dr. Jeff Tseng from the University of Oxford restructured the framework into more flexible and efficient C++ codes and implemented it into the SNO+ software\cite{jieMPW}. I first involved in testing and optimizing this reconstruction framework on the simulations and data.


The data taken from these statuses are used for background estimations. The framework is also extended for the scintillator phase and the Tellurium-loading phase as well. The principles and optimizations are described in Chapter 4.



In Chapter 2, 

neutrino physics

solar neutrinos
neutrino flavor transformation, neutrino mass, neutrinoless double beta decay.

In Chapter 3, the SNO+ experiment, how the SNO+ detector works and reads the physics data.
The properties of the liquid scintillator


In Chapter 5, this reconstruction framework is applied for the water phase physics analyses. I analyzed the calibration source data during the water phase. By comparing the simulations and data, I obtained the reconstruction resolutions and uncertainties. Then it is applied for the solar neutrino and backgrounds analysis.
The SNO+ official reconstruction results were compared with the and it shows that the better direction resolutions, which would be useful for the solar neutrino analyses.





Moving towards the Tellurium-loading phase for searching $0\nu\beta\beta$ process, the SNO+ collaboration started filling the liquid scintillator into the detector from July 2019. 

for background analysis.

These transition statuses are called "partial-fill".  

In Chapter 6, 

based on Monte Carlo simulations.

higher light yield

A light yield measurement for Tellurium-loaded liquid scintillator

The last chapter summarizes the works of this thesis.