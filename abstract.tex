This thesis presents a measurement of the Boron-8 ($^8$B) solar neutrinos. The measurement is based on a dataset of 190.33 live days acquired during the SNO+ water physics commissioning. To analyze the data, an event reconstruction framework was developed to evaluate the event vertex and direction. A multivariate analysis was applied to reduce the background events in the dataset. By analyzing the data within an energy range from 5 to 15 MeV, an observed elastic scattering flux assuming no neutrino flavor transformation is obtained as $\Phi_{\mathrm{ES}}=(2.07 \pm 0.218 (stats.)^{+0.12}_{-0.11}(syst.))\times10^6$ cm$^{-2}$s$^{-1}$ while the total $^8$B solar neutrino flux is evaluated as $\Phi_{\mathrm{^8B}}=(4.62 \pm 0.459 (stats.)^{+0.13}_{-0.08}(syst.))\times10^6$ cm$^{-2}$s$^{-1}$. These fluxes are consistent with the previous measurement done by SNO+\cite{anderson2019measurement}, and from the other experiments, such as Super Kamiokande\cite{abe2016solar}. The systematics were obtained by reconstructing and analyzing the calibration datasets from an Nitrogen-16 calibration source.

Currently, the SNO+ experiment has completed the water phase commission and is filled with the liquid scintillator. It turns from a Water Cherenkov detector into a 780-tonne liquid scintillator detector. Tellurium-130 isotopes will be loaded into the detector to fulfill the ultimate physics goal of SNO+: to search for the neutrinoless double beta decay. The other parts of this thesis discuss the reconstruction framework for the partial-fill and scintillator phases. For the scintillator phase, the event reconstruction gives a high position resolution down to about 65 mm. %A light yield measurement of the tellurium liquid scintillator is also presented in this thesis, which gives a relative light yield to the pure liquid scintillator as 0.65.