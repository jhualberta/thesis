\section{Abstract}

This thesis presents a measurement of the Boron-8 ($^8$B) solar neutrinos. The measurement is based on a dataset of 190.3 live days acquired during the SNO+ water physics commission. To analyze the data, an event reconstruction framework for evaluating the event vertex and direction was developed. A multivariate analysis was applied to reduce the background events in the dataset. Analyzing the events from the dataset within an energy range from 5 to 15 MeV, an observed flux of elastic scattering 

 
$\Phi_{ES}=(2.09 \pm 0.20(stats.)^{+0.06}_{-0.03}(stats.))\times10^6~cm^{-2}s^{-1}$. 
 
$\Phi_{^8B}=(4.48 \pm 0.44(stats.)^{+0.13}_{-0.08}(syst.))\times10^6~cm^{-2}s^{-1}$. The elastic scattering in this energy region is 


The systematics of the event reconstruction were obtained by analyzing the calibration datasets from an Nitrogen-16 calibration source. 






The SNO+ experiment will soon complete its commissioning and begin
searching for the neutrinoless double beta decay of tellurium, loaded within
its liquid scintillator. As a large-scale (780 tonne) liquid scintillator detector,
SNO+ will also be well positioned to make a precision measurement of antineutrinos, produced from nearby nuclear reactors. Measuring these antineutrinos
would provide direct information about the cores of these reactors and enable
a study of neutrino properties. In anticipation, a first search for antineutrinos
was performed over 69.7 live days of data collected while the SNO+ detector was filled with water, an intermediate commissioning phase. A combination of Monte Carlo simulations and measurements with radioactive calibration
sources were used to determine what the antineutrino signal (characterized by
a coincident positron and neutron) would look like in the detector.
The neutron modeling was first validated by performing a series of measurements of an americium-beryllium (AmBe) neutron source at the University of
Alberta. The neutron interactions were detected by irradiating various targets and measuring the γ rays of the resulting reactions using a High Purity
Germanium detector. Following Monte Carlo simulations of antineutrinos in
the SNO+ detector, a search algorithm was developed to distinguish this signal
ii
from naturally occurring backgrounds. Lastly, another $^{16}$N

 source was placed
within the SNO+ detector to exactly characterize its neutron detection capabilities. Searching the detector-collected data yielded a total of 5 antineutrino
candidate events in the region of interest. This was in agreement with expectations from another Monte Carlo simulation that was developed to model the
detector backgrounds for this specific signal. From this, an upper limit at 90%
confidence was determined for the flux of antineutrinos from nuclear reactors
passing through the SNO+ detector of (1.45 ± 0.23) x 107 ν¯/(cm2 s).


The SNO+ experiment has completed the water phase mission and is filled with the liquid scintillator.


is planned to explore one of the unknown properties of neutrinos: whether the neutrinos are Majorana particles or Dirac particles.

