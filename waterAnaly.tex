

During the August to October 2019, the PPO is added into the LAB when the water level at 5100 mm (in PSUP coordinate). This is for the SNO+ partial-fill phase.


\subsection{Sky-shine Classifier}
A ``sky shine'' (SkyShine) classifier was developed by the collaboration to discriminate 


The SkyShine classifier aims to discriminate "sky shine" events from other backgrounds by looking at the ratio of hit counts in a middle z range and a low z range. It can also look at neck and high-z OWL PMT hits. The classifier's behavior for partial fill is studied using simulation.

\cite{skyshine}


$\beta_{14}$ isotropy classifier
\[
\beta_l = \frac{2}{N(N-1)}\sum_{i=1}^{N-1}\sum_{j=i+1}^N P_l(\cos\theta_{ij})
\]

where $P_l(\cos\theta_{ij})$ are Legendre polynomials

$\beta_{14}=\beta_1+4\beta_4$


thetaij isotropy classifier describes the angle subtended at an event vertex by PMT \#i and PMT \#j.

\[
\cos\theta_{ij}=\frac{(\vec{X}_{PMT\#i}- \vec{X}_{event})\cdot (\vec{X}_{PMT\#j}- \vec{X}_{event})}{|\vec{X}_{PMT\#i}- \vec{X}_{event}||\vec{X}_{PMT\#j}- \vec{X}_{event}|}
\]