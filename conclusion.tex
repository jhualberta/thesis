\epigraph{Le vent se l\`{e}ve, il faut tenter de vivre.}{--- \textup{Paul Val\'{e}ry, \textit{Le cimeti\`{e}re marin}}}

In this thesis, a reconstruction algorithm framework (the Multi-Path Fitter) was developed for multiple SNO+ physics phases. For the SNO+ detector with a diameter of 12 m, it achieves position resolutions of 300~mm for 5 MeV electron event in the water phase and less than 70 mm for 2.5 MeV electron event in the scintillator phase. The position biases are within 100 mm for both phases. This framework has been applied to the SNO+ water phase and partial-fill phase analysis, and it has potentials to be applied in the scintillator phase and tellurium phase. This thesis used the fitter for analyzing the $^8$B solar neutrinos during the water phase. The fitter was also used by the SNO+ collaboration as the prime vertex fitter for the physics analyses during the partial-fill phase.

For the water phase, the reconstruction uncertainties in event position, direction and energy were determined by analyzing the data and simulations of $^{16}$N source calibrations. For a fiducial volume of 5.5 m, the position biases between the data and simulations in the $(x,y,z)$ axes are all within $\pm10$ mm; the fractional energy scale uncertainty is $1.0$\%, and the fractional uncertainty on the energy resolution at kinetic energy $T_e$ is $0.037\sqrt{T_e/\mathrm{MeV}}$.

By utilizing the water phase reconstruction, this thesis provides an alternative analysis for the $^8$B solar neutrino measurement during the SNO+ water phase. By looking at the low background dataset for 190.33 live days, a $^8$B solar neutrino rate of $0.953\pm0.0925$~events/(kt$\cdot$day) with a background rate of $0.310\pm 0.0607$ events/(kt$\cdot$day) in the energy region $[5,15]$ MeV were obtained. As the energy threshold is pushed down to 5 MeV, this background rate is still significantly low for the solar neutrino measurement in a water Cherenkov detector.

Based on the same dataset in the energy region $[5,15]$ MeV, an estimated $^8$B solar neutrino flux is evaluated as: 
\begin{equation*}
\Phi_{\mathrm{^8B}}=(4.62 \pm 0.447 \mathrm{(stat.)}^{+0.300}_{-0.137}\mathrm{(syst.)})\times 10^6 \mathrm{cm}^{-2}\mathrm{s}^{-1}\;, 
\end{equation*}
while the elastic scattering flux is:
\begin{equation*}
\Phi_{\mathrm{ES}}=(2.10 \pm 0.204\mathrm{(stat.)}^{+0.169}_{-0.0722}\mathrm{(syst.)})\times 10^6 \mathrm{cm}^{-2}\mathrm{s}^{-1}\;.
\end{equation*}

The $\Phi_{\mathrm{ES}}$ result given here is consistent with the recent measurements from the SNO+, Super-Kamiokande and Borexino experiments.