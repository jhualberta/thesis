

During the August to October 2019, the PPO is added into the LAB when the water level at 5100 mm (in PSUP coordinate). This is for the SNO+ partial-fill phase.


\subsection{Sky-shine Classifier}
A ``sky shine'' (SkyShine) classifier was developed by the collaboration to discriminate 



 


The SkyShine classifier aims to discriminate "sky shine" events from other backgrounds by looking at the ratio of hit counts in a middle z range and a low z range. It can also look at neck and high-z OWL PMT hits. The classifier's behavior for partial fill is studied using simulation.

\cite{skyshine}








$\beta_{14}$ isotropy classifier
\[
\beta_l = \frac{2}{N(N-1)}\sum_{i=1}^{N-1}\sum_{j=i+1}^N P_l(\cos\theta_{ij})
\]

where $P_l(\cos\theta_{ij})$ are Legendre polynomials

$\beta_{14}=\beta_1+4\beta_4$


thetaij isotropy classifier describes the angle subtended at an event vertex by PMT \#i and PMT \#j.

\[
\cos\theta_{ij}=\frac{(\vec{X}_{PMT\#i}- \vec{X}_{event})\cdot (\vec{X}_{PMT\#j}- \vec{X}_{event})}{|\vec{X}_{PMT\#i}- \vec{X}_{event}||\vec{X}_{PMT\#j}- \vec{X}_{event}|}
\]



\begin{tikzpicture}[node distance=2cm]
%%% settings
\tikzstyle{startstop} = [rectangle, rounded corners, minimum width=3cm, minimum height=1cm,text centered, draw=black, fill=red!30]
\tikzstyle{io} = [trapezium, trapezium left angle=70, trapezium right angle=110, minimum width=3cm, minimum height=1cm, text centered, draw=black, fill=blue!30]
\tikzstyle{process} = [rectangle, minimum width=3cm, minimum height=1cm, text centered, draw=black, fill=orange!30]
\tikzstyle{decision} = [diamond, aspect=2, minimum width=3cm, minimum height=1cm, text centered, draw=black, fill=green!30]
\tikzstyle{arrow} = [thick,->,>=stealth]
%%%%%%%%%%%%%%%%%%%

\node (start) [startstop] {Loop for events (loop 1, from first to last)};
\node (pro1) [process, below of=start] {Event \#i};
\node (dec1) [decision,below of=pro1, aspect=2.5, text width=6cm, yshift = -2cm ] {FV cut~\&~$Nhit_i\in(175,1700)$};
\node (loop2) [startstop, below of=dec1, yshift = -2cm ]{Loop for the events (loop 2, from i+1 to last)};
\node (pro2) [process, below of=loop2] {Event \#j};
\node (dec2) [decision, right of=pro2, text width=2.5cm, xshift =3cm] {$\Delta t=t_j -t_i>2000~\mu s$};
\node (pro3) [process, right of=loop2, text width=3cm, xshift =6.5cm] {break loop 2};
\node (dec3) [decision, aspect=2, below of=pro2, text width=5cm, yshift=-2.5cm] {FV cut \& $Nhit_j\in(175,320)$ \& $\Delta t\in(3.69,1000)~\mu s$ \& $|\vec{X}_i-\vec{X}_j|<0.5~m$};

\node (pro4) [process, right of=dec3, text width=4cm, xshift =7cm] {Record event pair: event \#i as prompt event, tagged as $^{214}$Bi; event \#j as delayed event, tagged as $^{214}$Po.};



\draw [arrow] (start) -- (pro1);
\draw [arrow] (pro1) -- (dec1);
\draw [arrow] (dec1)  -| node [anchor=east] {no} ([xshift=-1cm] dec1.west)
|- ([yshift=+1cm] start.north) coordinate (aux)-- (start.north);
\draw [arrow] (dec1) -- node[anchor=east] {yes}(loop2);
\draw [arrow] (loop2) -- (pro2);
\draw [arrow] (pro2) -- (dec2);
\draw [arrow] (dec2) -|node [anchor=west] {no} (pro3);
\draw [arrow] (pro3.north)|-(start.east);
\draw [arrow] (dec2.south) node[anchor=north]{yes}-|(dec3.north);
\draw [arrow] (dec3.east)--node[anchor=south] {yes}(pro4.west);
\draw [arrow] (dec3.west)node[anchor=east] {no}|-(loop2.west);
\draw [arrow] ([xshift=1.5cm] pro4.north)|-(pro3.east);
\end{tikzpicture} 





$\phi=\hat{\vec{v}}_e\cdot\hat{\vec{v}}_{assume}$

$\vec{n}=\hat{\vec{v}}_e\times\hat{\vec{v}}_{assume}$
\[
R=\begin{bmatrix}
	n^2_x(1-\cos\phi)+\cos\phi       &n_xn_y(1-\cos\phi)-n_z\sin\phi & n_xn_z(1-\cos\phi)+n_y\sin\phi \\
	n_xn_y(1-\cos\phi)+n_z\sin\phi & n^2_y(1-\cos\phi)+\cos\phi & n_yn_z(1-\cos\phi)-n_x\sin\phi \\
	n_xn_z(1-\cos\phi)-n_y\sin\phi & n_yn_z(1-\cos\phi)+n_x\sin\phi & n^2_z(1-\cos\phi)+\cos\phi
\end{bmatrix}
\]
$\vec{v'}_e=R\vec{v}_e$

$\vec{X'}_{evt}=R\vec{X}_{evt}$

Move $\vec{X'}_{evt}$ to the origin,

$\vec{X'}_{pmt}=R\vec{X}_{pmt}-\vec{X'}_{evt}$

Breit-Wigner function
\[
p(x) = \frac{c_0}{\pi}\frac{\frac{1}{2} \Gamma}{(x-m)^2 + (\frac{1}{2} \Gamma)^2}+c_1
\]