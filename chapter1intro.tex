%\setlength{\epigraphwidth}{0.5\textwidth}
%\epigraph{``Even so," said Trevize, ``I must search. Even if the endless powdering of stars in the Galaxy makes the quest seem hopeless, and even if I must do it alone."}{--- \textup{Isaac Asimov}, \textit{Foundation and Earth }}

Among state-of-the-art neutrino experiments, SNO+ aims to search for an extremely rare process called neutrinoless double-beta decay ($0\nu\beta\beta$). This search will explore and attempt to resolve a key question as to the nature of the neutrino: is it a Majorana or a Dirac particle. A discovery that  the $0\nu\beta\beta$ process occurs would unravel the masses of neutrinos and test new physics theories.

Regarding its physics targets, the SNO+ experiment goes through three major stages, mainly determined by the working medium inside the SNO+ detector. First, the detector was filled with water and operated as a water Cherenkov detector. During this water phase, over 300 live days of data have been collected. Calibration sources were also deployed for measuring the detector properties. Based on the first 114.7 live days of data, SNO+ published a measurement of the fluxes of Boron-8 neutrinos from the Sun. The results are consistent with the other solar neutrino experiments and also demonstrate  extremely low background levels for the analysis\cite{anderson2019measurement}.

As of the time of writing this thesis, the water phase stage has been completed, and liquid scintillator has replaced the water. During the scintillator filling, there were several long intervals during which the water/scintillator interface level remained stable at a fixed height ($z$) inside the detector. Collectively these stable (albeit) transitional stages are named the ``partial-fill phase'', during which detector data are used to analyze the properties and estimate the backgrounds of the liquid scintillator for the two physics phases to follow.

Once the detector is fully filled with the liquid scintillator, the detector will be operated during the ``scintillator phase''. A 6-month data-taking interval is planned, to establish the background levels of the liquid scintillator and to measure solar neutrinos, reactor antineutrinos and geoneutrinos\cite{directorReview}. After this scintillator phase, tellurium isotopes will be loaded into the liquid scintillator and once the mixture (or ``cocktail'') is stable, the search for the $0\nu\beta\beta$ signal will commence.

Event reconstruction is crucial for the physics analysis. In this thesis, a framework of reconstruction algorithm, called the ``multiple-path fitter'' (\texttt{MP fitter}), was developed for multiple SNO+ physics phases. This framework was first developed by the author's supervisor, Dr. A. L. Hallin, to reconstruct and investigate the data taken during the ``partial-fill water'', which was an early stage of the experiment when the detector was only {\em partially} filled with water (the residual volume being air) in December, 2014\cite{partialWater}. Drs. K. Singh and D. J. Auty (U. Alberta) further developed this fitter to accommodate the wavelength shifter and analyse water events (\cite{davidPartialWater, kalpanaWLS, kalpanaWLS2, kalpanaMPFitter}), while Dr. J. Tseng (U. Oxford) restructured the framework using more flexible and efficient C++ code logic, and implemented it into the SNO+ software\cite{jieMPW}. I was first involved in testing and optimizing the \texttt{MP fitter} on simulations and data. Then I extended its usage by developing an \texttt{MP partial fitter} for the partial-fill phase and an \texttt{MP scint fitter} for both the scintillator phase and tellurium phase. With these extensions, the \texttt{MP fitter} framework is ready for multiple SNO+ physics phases. The principles, optimizations, and performances of these fitters are described in Chapter 4 and Appendix A. The key research results presented in this thesis stem from application of the \texttt{MP water fitter} to calibration data taken during the SNO+ water phase and to the 190.3 live days of water phase data. Based on these data, a measurement of the Boron-8 solar neutrino flux was performed.

The thesis is organised as follows. In Chapter 2, basic neutrino properties and the phenomena of neutrino flavor transformation and neutrinoless double beta decay are introduced, along with the relevant theories and experiments. Chapter 3 is an overview of the SNO+ experiment, covering how the SNO+ detector works and reads the physics data; optical properties of liquid scintillators and the detector calibrations, which are crucial to the reconstruction; and a bench-top light yield measurement for a tellurium-loaded liquid scintillator. The latter measurement shows a light yield shift due to the humidity content of the tellurium-loaded scintillator in addition to (an expected) light yield sensitivity to tellurium concentration. Since the light yield of the scintillator is crucial to event reconstruction, this study is helpful for the detector running in realistic situations.

As mentioned above, Chapter 4 describes the principle of event reconstruction and focuses on the \texttt{MP fitter}. The other reconstruction algorithms, for example the energy reconstruction, are also introduced. Chapter 5 focuses on the calibration during the SNO+ water phase. The \texttt{MP water fitter} was applied to the calibration data and simulations. Among the reconstructed quantities, the position and direction results were based on the MultiPath water fitter, while the energy and classifier results were extracted using the SNO+ official algorithms. However, these results (energy, event type) depend on the position and direction results provided by \texttt{MP water fitter}. By comparing simulations and data, I obtained the reconstruction resolutions and uncertainties, following procedures suggested by the collaboration. This chapter also discusses the calibration during the partial-fill phase. The \texttt{MP partial fitter} was applied. Based on the calibration data, analysis for extracting the Cherenkov signals from the scintillation lights is discussed.

The results of Chapter 5 underpin an analysis (in Chapter 6) of solar neutrinos during the SNO+ water phase. The \texttt{MP water fitter} was applied to the water phase physics data and simulations. Based on the simulations, I applied a machine learning analysis to optimize the signal and background separation. Then the optimized separation parameters were applied to the data to extract the solar neutrinos from the backgrounds. I evaluated the solar neutrino rates and the background rates from the dataset. The systematics and uncertainties from Chapter 5 were evaluated and included in the results. Finally, a $^8$B solar neutrino flux was evaluated.

Note: unless otherwise stated or cited in the text, the analyses of simulations and data from Chapters 4 to 6 are my own work, performed under the supervision of Drs. A. L. Hallin, J.-P. Ya\~{n}ez Garza, and C. B. Krauss.
