The results described in this thesis required the effort of many individuals from the multi-national SNO+ collaboration. While the thesis focuses on the work performed by the author, some of the work 

was performed
with and are using tools developed by the researchers involved.


Note: unless otherwise stated or cited in the text, the analyses of simulations and data from Chapters 4 to 6 are my own work, performed under the supervision of Drs. A. L. Hallin, J. P. Y\'{a}\~{n}ez Garza, and C. B. Krauss. Usage of work other than the author's is appropriately cited in
the text. 

the software programs developed by the SNO+ collaboration.

SNO+ working groups

Chapter two is a review of the literature
and experimental results of neutrino physics available at the time of writing this work. 

Chapter three is an overview of the SNO+ detector based on the work of the SNO+ collaboration. The principles of detection 

are literature review. 


The light yield measurements presented in the same chapter was performed by the author, with the assistance of the author's supervisor, Dr. A. L. Hallin, and Dr. M. Sharma and Prof. J. Veinot from the Chemistry Department at U. Alberta.

The framework of the reconstruction algorithm presented in the Chapter 4 was first developed by Dr. A. L. Hallin. Drs. K. Singh and D. J. Auty (U. Alberta) further developed and extended the framework. Dr. J. Tseng (U. Oxford) restructured the framework using more flexible and efficient C++ code logic, and implemented it into the SNO+ software. I was first involved in testing and optimizing the algorithm on simulations and data. Then I extended its usage by developing an \texttt{MP partial fitter} for the partial-fill phase and an \texttt{MP scint fitter} for both the scintillator phase and tellurium phase. 


The key research results presented in this thesis stem from application of the \texttt{MP water fitter} to calibration data taken during the SNO+ water phase and to the 190.3 live days of water phase data. Based on these data, a measurement of the Boron-8 solar neutrino flux was performed.

The majority of the work in Chapter 4

In this thesis, a framework of reconstruction algorithm, called the ``multiple-path fitter'' (\texttt{MP fitter}), was developed for multiple SNO+ physics phases. This framework was first developed by the author's supervisor, Dr. A. L. Hallin, to reconstruct and investigate the data taken during the ``partial-fill water'', which was an early stage of the experiment when the detector was only {\em partially} filled with water (the residual volume being air) in December, 2014\cite{partialWater}. Drs. K. Singh and D. J. Auty (U. Alberta) further developed this fitter to accommodate the wavelength shifter and analyse water events (\cite{davidPartialWater, kalpanaWLS, kalpanaWLS2, kalpanaMPFitter}), while Dr. J. Tseng (U. Oxford) restructured the framework using more flexible and efficient C++ code logic, and implemented it into the SNO+ software\cite{jieMPW}. I was first involved in testing and optimizing the \texttt{MP fitter} on simulations and data. Then I extended its usage by developing an \texttt{MP partial fitter} for the partial-fill phase and an \texttt{MP scint fitter} for both the scintillator phase and tellurium phase. With these extensions, the \texttt{MP fitter} framework is ready for multiple SNO+ physics phases. The principles, optimizations, and performances of these fitters are described in Chapter 4 and Appendix A. The key research results presented in this thesis stem from application of the \texttt{MP water fitter} to calibration data taken during the SNO+ water phase and to the 190.3 live days of water phase data. Based on these data, a measurement of the Boron-8 solar neutrino flux was performed.

Chapter 5 focuses on the calibration during the SNO+ water phase. The \texttt{MP water fitter} was applied to the calibration data and simulations. Among the reconstructed quantities, the position and direction results were based on the MultiPath water fitter, while the energy and classifier results were extracted using the SNO+ official algorithms. However, these results (energy, event type) depend on the position and direction results provided by \texttt{MP water fitter}. By comparing simulations and data, I obtained the reconstruction resolutions and uncertainties, following procedures suggested by the collaboration. This chapter also discusses the calibration during the partial-fill phase. The \texttt{MP partial fitter} was applied. Based on the calibration data, analysis for extracting the Cherenkov signals from the scintillation lights is discussed.

The results of Chapter 5 underpin an analysis (in Chapter 6) of solar neutrinos during the SNO+ water phase. 

The \texttt{MP water fitter} was applied to the water phase physics data and simulations. Based on the simulations, I applied a machine learning analysis to optimize the signal and background separation. Then the optimized separation parameters were applied to the data to extract the solar neutrinos from the backgrounds. I evaluated the solar neutrino rates and the background rates from the dataset. The systematics and uncertainties from Chapter 5 were evaluated and included in the results. Finally, a $^8$B solar neutrino flux was evaluated.

 The SNOMAN Monte-Carlo used in the analysis is
the work of the SNO 
ollaboration.




 using the computing resources, provided by Compute Canada
 
 and also with the helps of technical support team at U. Alberta.
 
 data with the efforts