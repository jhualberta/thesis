Since the work in this thesis relate to the SNO+ experiment, the results described herein required the effort of many individuals from the multi-national SNO+ collaboration. While the thesis focuses on the original work performed by the author, some of the work was based on the ideas, methods or tools provided by the SNO+ collaboration.

Usage of work other than the author's is appropriately cited in the text. The theoretical and experimental contents about neutrino presented in Chapter 2 are a review of the literature and results available at the time of writing this thesis.

Chapter 3 is an overview of the SNO+ detector based on the work of the collaboration. The relative light yield measurements of the Te-loaded liquid scintillators presented in Sect.~\ref{sect:RelativeLightYieldMeasurement} was performed by the author, with the assistance from the author's supervisor, Dr. A. L. Hallin, and Dr. M. Sharma and Prof. J. Veinot from the Chemistry Department at U. Alberta.

Unless otherwise stated or cited in the text, the analyses of simulations and data from Chapters 4 to 6 are my own work, performed under the supervision of Drs. A. L. Hallin, J. P. Y\'{a}\~{n}ez Garza, and C. B. Krauss. 

The framework of the reconstruction algorithms presented in the Chapter 4 was first developed by Dr. A. L. Hallin. Drs. K. Singh and D. J. Auty (U. Alberta) further developed and extended the framework for the SNO+ water phase. The SNO+ reconstruction and software working groups, as well as the Code Integrity Committee helped to implement the framework into the SNO+ analysis software (\texttt{RAT}). The author involved in testing and optimizing the algorithms on simulations and data. The author also extended the framework's usage for multiple SNO+ physics phases. Particularly, the author developed the position reconstruction algorithm for the SNO+ partial-fill phase. The algorithm requires the parameters which were measured and determined by the collaboration. To develop the reconstruction framework, the author performed simulations by using the \texttt{RAT}. The studies on the simulations of the wavelength-shifter described in Sect.~\ref{sect:WLSfitter} was performed by the author, while the reconstruction algorithm was developed by Dr. K. Singh. The other reconstruction algorithms developed by the collaboration were also introduced briefly in this chapter. 

The data used in Chapters 5 and 6 were collected by the SNO+ detector during the operation, which require a collaborative effort. The simulations used in the two chapters were mostly produced by the SNO+ calibration, background and simulation working groups. The author applied the reconstruction algorithms (described in chapter 4) to the data and simulations, and then evaluated the reconstruction systematics for the water physics in Chapter 5, following the routines and methods provided by the SNO+ water physics and analysis working groups.

In Chapter 6, the author used the position and direction reconstruction algorithms to analyze the water physics data. A separation of signal and background based on machine learning was performed by the author. The author evaluated the Boron-8 solar neutrino rates and the background rates from the dataset. The systematics and uncertainties from Chapter 5 were evaluated and included in the results by the author.

The author executed algorithms on data and simulation mentioned above by using the Compute Canada computing resources, that were allocated to Dr. C. B. Krauss, and also with the help of the technical support team at U. Alberta.