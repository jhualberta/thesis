Since the research in this thesis relates to the SNO+ experiment, the results described herein required the effort of many individuals from the multi-national SNO+ collaboration. While the thesis focuses on the original work performed by the author, some of the research was based on the ideas, methods, or tools provided by the SNO+ collaboration.

Usage of work other than the author's is appropriately cited in the text. The theoretical and experimental results and discussions presented in Chapter 2 are a review of the literature at the time of writing this thesis. Chapter 3 is an overview of the SNO+ detector based on the work of the collaboration, along with literature reviews. There is one exception in Chapter 3: the relative light yield measurements of the Te-loaded liquid scintillators were performed by the author, with the assistance of the author's supervisor, Dr. A. L. Hallin. Dr. M. Sharma and Prof. J. Veinot from the Department of Chemistry at the University of Alberta provided the samples for the measurement.

Unless otherwise stated or cited in the text, the analyses of simulations and data from Chapters 4 to 6 are the author's own work, performed under the supervision of Drs. A. L. Hallin, J. P. Y\'{a}\~{n}ez Garza, and C. B. Krauss.

The framework of the reconstruction algorithms presented in Chapter 4 was first developed by Dr. A. L. Hallin. Drs. K. Singh and D. J. Auty at the University of Alberta further developed and extended the framework for the SNO+ water phase. The SNO+ reconstruction and software working groups, as well as the Code Integrity Committee (CIC), helped to implement the framework into the SNO+ analysis software (\texttt{RAT}). The author was responsible for testing and optimizing the algorithms on simulations and data, as presented in this chapter. The author also extended the framework's usage for multiple SNO+ physics phases, particularly for the SNO+ partial-fill phase. The algorithm requires the parameters which were measured and determined by the collaboration. To develop the reconstruction framework, the author performed simulations by using the \texttt{RAT}. The studies on the simulations of the wavelength-shifter were performed by the author, while the reconstruction algorithm was developed by Dr. K. Singh. The other reconstruction algorithms developed by the collaboration were also introduced briefly in Chapter 4.

The members of the SNO+ calibration working group deployed the $^{16}$N source for the calibration mentioned in Chapter 5. The data used in Chapters 5 and 6 were collected by the SNO+ detector during the operation, which requires a collaborative effort. The simulations used in these two chapters were mostly produced by the SNO+ calibration, background, and simulation working groups. The author applied the reconstruction algorithms (described in Chapter 4) to the data and simulations, and then evaluated the reconstruction systematics for the water physics in Chapter 5, following the routines and methods provided by the SNO+ water physics and analysis working groups.

In Chapter 6, the author analyzed the water physics data by using the reconstruction algorithms mentioned in Chapter 4. The separation of signal and background based on machine learning was performed by the author. The author evaluated the Boron-8 solar neutrino flux, and the signal and background rates from the dataset. The systematics and uncertainties from Chapter 5 were evaluated and included in the results by the author.

The author executed algorithms on data and simulation mentioned above by using the Compute Canada computing resources, that were allocated to Dr. C. B. Krauss. The author also received assistance for using the resources from the technical support team at the University of Alberta.