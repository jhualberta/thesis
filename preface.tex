%\setlength{\epigraphwidth}{0.5\textwidth}
%\epigraph{``Even so," said Trevize, ``I must search. Even if the endless powdering of stars in the Galaxy makes the quest seem hopeless, and even if I must do it alone."}{--- \textup{Isaac Asimov}, \textit{Foundation and Earth }}

Among state-of-the-art neutrino experiments, SNO+ aims to search for an extremely rare process called neutrinoless double-beta decay ($0\nu\beta\beta$). This search will explore the unknown nature of neutrinos: whether they are Majorana or Dirac particles, unravel the masses of neutrinos, and test the new physics theories as well.

The SNO+ experiment goes through three major stages, mainly determined by the working medium inside the SNO+ detector. First, the detector was filled with water and worked as a water Cherenkov detector. During this water phase stage, over 300 live days of data have been collected to measure fluxes of neutrinos from the Sun.

Rare decay process modes, the invisible nucleon decay predicted by the Grand Unified Theory (GUT), have also been searched. 

During this phase, calibration sources have been deployed for measuring the detector properties. Up until this thesis writing, the water phase stage has been completed, and the liquid scintillator has replaced the water. During the scintillator filling, there were several long periods when the water level was stable at a fixed height. 

These stable transition stages are called the ``partial-fill phase'', which are used to estimate the backgrounds of the liquid scintillator for the next physics phases.

Once the detector is fully filled with the liquid scintillator, the experiment will walk into the ``scintillator phase''. A 6-month data-taking is planned for counting the backgrounds of pure liquid scintillator and do measurements of solar, reactor, and geo-neutrinos as well\cite{directorReview}. 

After the scintillator phase, tellurium isotopes will be loaded into the liquid scintillator. Once the mixtures are stable, the experiment will start searching for the neutrinoless double-beta decay signal.

In this thesis, a framework of reconstruction algorithm, called the ``multiple-path fitter'' (\texttt{MP fitter}), was developed for multiple SNO+ physics phases. This framework was first developed by the author's supervisor, Dr. Aksel Hallin, to reconstruct and investigate the data taken at an early stage of SNO+ when the detector was partially filled with water at the end of 2014 (called ``partial water fill stage'')\cite{partialWater}. Dr. Kalpana Singh and Dr. David Auty from the University of Alberta further developed it for investigating the wavelength shifter and water events\cite{davidPartialWater, kalpanaWLS, kalpanaWLS2, kalpanaMPFitter}. Dr. Jeff Tseng from the University of Oxford restructured the framework by using more flexible and efficient C++ code logic and then implemented it into the SNO+ software\cite{jieMPW}.  I was first involved in testing and optimizing the \texttt{MP fitter} on the simulations and data. Then I further developed it for multiple SNO+ physics phases, i.e., to extend its usage for the partial-fill phase (making a \texttt{MP partial fitter}), scintillator phase, and tellurium phase (making a \texttt{MP scint fitter} for these two phases). The principles, optimizations, and performances of these fitters are described in Chapter 4 and Appendix A.

Chapter 2 introduces neutrino physics. This chapter focuses on the phenomena of neutrino flavor transformations and neutrinoless double beta decay. The relevant theories and experiments are introduced.

Chapter 3 introduces the SNO+ experiment. This chapter explains how the SNO+ detector works and reads the physics data. The chapter also shows the optical properties of liquid scintillators and the calibrations of the detector, which are crucial to the reconstruction. A bench-top light yield measurement for a tellurium-loaded liquid scintillator is also discussed in this chapter. This measurement shows a light yield shift due to the humidity exposure of the tellurium-loaded scintillator as well as a change of concentrations of tellurium. Since the light yield of the scintillator is crucial to the event reconstruction, this study is helpful for the detector running in realistic situations.

As mentioned previously, Chapter 4 describes the principle of event reconstruction and focuses on the \texttt{MP fitter}. The other reconstruction algorithms, for example the energy reconstruction, are also introduced. 

Chapter 5 focuses on the calibration during the SNO+ water phase. The \texttt{MP water fitter} was applied to the calibration data and simulations. Among the reconstructed quantities, the position and direction results were from the MultiPath water fitter, while the energy and classifier results were from the SNO+ official algorithms. However, these results depend on the position and direction results. By comparing the simulations and data, I obtained the reconstruction resolutions and uncertainties, following the procedures suggested by the collaboration. These results were applied to the solar neutrino analysis in the next chapter. This chapter also discusses the calibration during the partial-fill phase. The \texttt{MP partial fitter} was applied. Based on the calibration data, analysis for extracting the Cherenkov signals from the scintillation lights is discussed.

In Chapter 6, I performed an analysis of solar neutrinos during the SNO+ water phase. The \texttt{MP water fitter} was applied to the water phase physics data and simulations. Based on the simulations, I applied a machine learning analysis to optimize the signal and background separation. Then the optimized separation parameters were applied to the data to extract the solar neutrinos from the backgrounds. I evaluated the solar neutrino rates and the background rates from the dataset. The systematics and uncertainties from Chapter 5 were evaluated and included in the results. Finally, a $^8$B solar neutrino flux was evaluated.

The last chapter summarizes the work and results of the thesis.