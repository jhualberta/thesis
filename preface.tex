Since the researches in this thesis relate to the SNO+ experiment, the results described in this thesis required the effort of many individuals from the multi-national SNO+ collaboration. While the thesis focuses on the work performed by the author, some of the work was based on the methods and tools developed by the SNO+ working groups. 

Usage of work other than the author's is appropriately cited in the text. Chapter 2 is a review of the literature and experimental results of neutrino physics available at the time of writing this work. 

Chapter 3 is an overview of the SNO+ detector based on the work of the SNO+ collaboration. The relative light yield measurements of the Te-loaded liquid scintillators presented in the same chapter (Sect.~\ref{sect:RelativeLightYieldMeasurement}) was performed by the author, with the assistance from the author's supervisor, Dr. A. L. Hallin, and Dr. M. Sharma and Prof. J. Veinot from the Chemistry Department at U. Alberta.

Unless otherwise stated or cited in the text, the analyses of simulations and data from Chapters 4 to 6 are my own work, performed under the supervision of Drs. A. L. Hallin, J. P. Y\'{a}\~{n}ez Garza, and C. B. Krauss. 

The framework of the reconstruction algorithm presented in the Chapter 4 was first developed by Dr. A. L. Hallin. Drs. K. Singh and D. J. Auty (U. Alberta) further developed and extended the framework for the SNO+ water phase. Dr. J. Tseng (U. Oxford) restructured the framework using more flexible and efficient C++ code logic, and implemented it into the SNO+ software. The author first involved in testing and optimizing the algorithm on simulations and data. To do these, the author used the software programs developed by the SNO+ collaboration for the detector simulation (\texttt{RAT}) and basic tools for accessing data (\texttt{RAT Tool}). Later, the author extended the framework's usage for the SNO+ multiple physics phases, which is crucial for the analyses during the SNO+ partial-fill phase. The reconstruction algorithms relay on the data and parameters measured by the bench-top experiments. These measurements were performed by the collaboration.

The various simulations and data presented in Chapters 5 and 6, 

collected using the SNO+
detector, were the result of a collaborative effort in detector operation. 


The
	design of an antineutrino search algorithm (Chapter 5), its application on the
	data and simulations (Chapters 5{8), and its optimization (Chapter 6) were
		my own work.




The author applied the reconstruction algorithm 


Chapter 5 focuses on the calibration during the SNO+ water phase. The \texttt{MP water fitter} was applied to the calibration data and simulations. Among the reconstructed quantities, the position and direction results were based on the MultiPath water fitter, while the energy and classifier results were extracted using the SNO+ official algorithms. However, these results (energy, event type) depend on the position and direction results provided by \texttt{MP water fitter}. By comparing simulations and data, I obtained the reconstruction resolutions and uncertainties, following procedures suggested by the collaboration. This chapter also discusses the calibration during the partial-fill phase. The \texttt{MP partial fitter} was applied. Based on the calibration data, analysis for extracting the Cherenkov signals from the scintillation lights is discussed.




The simulations of the SNO+ detector, described in Chapters 5, were performed by myself using the software programs developed by the SNO+ collaboration. The background simulation, also detailed in Chapter 5 was my own
work. 









The results of Chapter 5 underpin an analysis (in Chapter 6) of solar neutrinos during the SNO+ water phase. 

The \texttt{MP water fitter} was applied to the water phase physics data and simulations. Based on the simulations, I applied a machine learning analysis to optimize the signal and background separation. Then the optimized separation parameters were applied to the data to extract the solar neutrinos from the backgrounds. I evaluated the solar neutrino rates and the background rates from the dataset. The systematics and uncertainties from Chapter 5 were evaluated and included in the results. Finally, a $^8$B solar neutrino flux was evaluated.

The \texttt{RAT} Monte-Carlo used in the analysis is the work of the SNO collaboration.

using the computing resources, provided by Compute Canada

and also with the helps of technical support team at U. Alberta.

data with the efforts


The author also got a few suggestions and ideas from Brian Karr and Dr. Alex Wright.
