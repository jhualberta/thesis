%%% chapter of SNO+ detector
%\setlength{\epigraphwidth}{0.5\textwidth}
%\epigraph{In all experimental science the techniques for obtaining measurements are almost as important as the measurements themselves.}{--- \textup{J. D. Bernal}, \textit{The Social Function of Science}}

\section{Overview}\label{sect:overview}
The SNO+ experiment is located at SNOLAB. This deep underground facility sits at Vale's Creighton mine in Sudbury, Ontario, Canada (coordinate: 46$^\circ$28'19.6"N, 81$^\circ$11'12.4"W). It provides an environment with extremely low cosmic ray backgrounds. At sea level, an average cosmic muon ($\mu$) flux rate is about $1.44\times 10^7~\mu/m^2/day$l\cite{muonflux}. Cosmic muons with high energies (mostly $\mathcal{O}$(GeV)) can induce spallation backgrounds, such as fast neutrons and lasting isotopes, which are harmful to the low background counting experiments\cite{beacom2017physics}. The SNOLAB has a $2092\pm6$ m flat overburden of rock, which is $5890\pm94$ water equivalent meter (m.w.e). This rock overburden ensures that cosmic muon ($\mu$) flux rate is as low as $0.286\pm0.009~\mu/m^2/day$\cite{snop_nim}, which means that every hour only about 1 $\mu$ passes through the SNO+ detector. 

The SNO+ detector is a refurbishment of the SNO detector. The SNO+ collaboration makes use of the SNO infrastructure and upgraded it to be a liquid scintillator detector. As shown in Fig.~\ref{snopdetector}, the detector is inside a barrel-like rock cavity with a diameter of 22 m at its waist and a
height of 34 m. The cavity is filled with 7000 tonnes of ultrapure water (UPW) to provide both buoyancy for the vessel and radiation shielding of the surrounding backgrounds, such as the cosmic rays and isotope decays from the rock. 

The detector consists of an acrylic vessel (AV) sphere of 12.01 m in diameter and 5.5 cm in thickness. The AV contains detection medium and is held in place by a rope net system including hold-up and hold-down Tensylon ropes. This spheric structure is simple in geometry and reduces the complexities for simulation and event reconstruction. Furthermore, this geometry allows for spherical fiducial volume cuts from the center of the AV to further get rid of external backgrounds, which makes the SNO+ as a graded-shield type detector\cite{waterfield2017optical}.
On the top of the AV sphere, there is an acrylic neck cylinder with 6.8 m high and 1.46 m inner diameter. The neck connects the AV sphere to the facilities on the deck above the detector. Through the neck, pipes can fill detection medium into the AV and recirculate as well. Calibration sources for internal scans can also be lowered down into the AV through the neck.

The AV sphere is concentric within a stainless steel photomultiplier (PMT) support structure (PSUP), which is a geodesic dome with an average radius of 8.4 m. Hamamatsu 8-inch R1408 PMTs were mounted on the PSUP and 9394 PMTs are looking inward to the AV. A 27 cm diameter concentrator consisting pedals coated with aluminum was mounted on each PMTs to increase their light collection efficiency as well as the photocathode coverage of the detector, which reaches about 54\% effective coverage. Besides the inward-looking PMTs, 90 PMTs are looking outward, serving as muon vetos.  Four Hamamatsu R5912 High Quantum Efficiency (HQE) PMTs were also installed for testing the performance of potential SNO+ phases-II\cite{stringer2019sensitivity}. 

\begin{figure}[htbp]
	\centering
	\includegraphics[width=10cm]{SNOPdetector.png}
	\caption{The SNO+ detector labelled with main structures, figure modified from \cite{jones2011background}.}
	\label{snopdetector}
\end{figure}

\section{SNO+ Physics Phases}
The SNO+ experiment is designed for multi-purpose measurements of neutrino physics. The detector has been running since December 2016. There are three physics phases of the experiment and each phase has different detection medium inside the AV: the water phase, the scintillator phase and the tellurium phase\cite{whitepaper}. 

\subsubsection{Water Phase} 
In this initial phase, the AV was filled with about 905 tonnes of ultra pure water (UPW). The detector collected water physics data from May 2017 to July 2019.

During the data-taking, different types of calibration runs have been taken. The detector timing and energy response, systematics and backgrounds are studied. Multiple physics analyses of invisible nucleon decay, solar neutrinos and reactor antineutrinos are ongoing and a few results have been published\cite{anderson2019search,anderson2019measurement,anderson2020measurement}. The external backgrounds are also measured, which will be the same as the following two phases.

The main physics goal in this phase is to search for the invisible nucleon decay, which violates baryon number and is a prediction of Grand Unified Theory (GUT). In the invisible decay mode, a proton or a bound neutron decays away without releasing charged particles, compared to the ``visible'' decay channels of $p\to e\pi$ and $p\to\nu K$ which has been searched and set limits by the SuperK experiment. In the SNO+ water detector, $^{16}$O may decay into $^{15}$O$^*$ (bound neutron invisible decay) or $ ^{15}$N$^*$ (proton invisible decay) excited state. The $^{15}$O$^*$ has 44\% chance to deexcite to produce 6.18 MeV $\gamma$ ray and 2\% chance to produce 7.03 MeV $\gamma$; while $^{15}$N$^*$ has 41\% to release 6.32 MeV $\gamma$ and 7.01, 7.03 and 9.93 MeV $\gamma$ with chances of 2\%, 2\% and 3\% respectively. The experiment has searched for these $\gamma$ signals and the first publication sets world-leading limits of $\mathcal{O}(10^{29})$ years for both the proton and neutron invisible decay lifetime at 90\% Bayesian credibility level\cite{anderson2019search}. 

The $^8$B solar neutrinos were measured with a 69.2 kilo-tonnes$\cdot$day dataset. By analyzing the solar neutrino elastic scattering events based on the dataset (the method will be discussed in Chapter 5 with more details), the number of the solar neutrino events were counted in different energy regions. In the first publication\cite{anderson2019measurement}, by fitting to the non-oscillation solar neutrino model, an observed flux was obtained from the dataset to be $2.53^{+0.31}_{-0.28}(stat.)^{+0.13}_{-0.10}(syst.)\times 10^6$ cm$^{-2}$s$^{-1}$ for the $^8$B solar neutrinos with energies larger than 5 MeV. In the energy region larger than 6 MeV, the dataset has a background rate of $0.25^{+0.09}_{-0.07}$ events/kilo-tonne$\cdot$day, which is extremely pure with solar neutrinos. Currently, this background rate is the lowest one compared to the other water Cherenkov detectors\cite{anderson2019measurement}. 

Reactor antineutrinos can be captured by the SNO+ detector and get measured. 40\% of these antineutrinos are from one nearby reactor complex in Canada at a 240-km baseline; 20\% are from two Canadian complexes at around 350 km; the rest are from USA and elsewhere with longer baselines\cite{whitepaper}. Though in the pure water the antineutrino event rate is much smaller than in the scintillator, during the water phase, SNO+ still has potentials to detect reactor antineutrinos since the low background dataset and relatively high detection efficiency. To evaluate the sensitivity for detecting reactor antineutrinos, an Americium-Beryllium (AmBe) calibration source was deployed during the water phase. This source provides neutrons along with 4.4-MeV $\gamma$s. Neutrons are captured by hydrogen and 2.2-MeV $\gamma$s are released. An analysis of delayed coincidence between 4.4-MeV and 2.2-MeV $\gamma$s can help to tag neutrons, which is crucial for tagging the reactor antineutrinos since they are measured by the inverse $\beta$-decay process which produces neutrons 
with similar energy scale. In the first publication for the SNO+ water phase, a neutron detection efficiency of 50\% was obtained when the AmBe source was deployed at the centre of the detector; the neutron-hydrogen capture time constant $\tau$ was measured to be $202.35_{-0.76}^{+0.87}~\mu s$ and from $\tau$, a thermal capture cross section was calculated to be $336.3^{+1.2}_{-1.5}$ milli-barn (mb)\cite{anderson2020measurement}.

%%%%%%% top neck fill, partial Fill????
%The SNO+ water data taking from May 2017 to September 2018. The period from May 2017 to October 2018 is the first stage of the water phase. During this stage, several calibration runs were taken, including the $^{16}$N calibration scans and the laserball scans. During the period from October 2018 to July 2019, over 20 tonnes of LAB (without PPO) was filled into the detector and the liquid mostly occupied the neck volume, slightly below the neck bottom. With the nitrogen cover gas on the top, the data taken during this period is considered as low background data.
%
%
%This partial filled transition phase (called ``partial-fill'' phase) is mainly aimed to understand the in-situ backgrounds of scintillator. 
%A six to twelve months of data-taking are expected for this phase. 
%During the filling, the partially-filled detector has been operated at several different water level (mainly at 5.7 m, 0.75 m)

\subsubsection{Scintillator Phase}

In this phase, the AV will be filled with 780 tonnes of liquid scintillator, which is a mixture of linear alkylbenzene (LAB) as a solvent and 2 g/L of 2,5-diphenyloxazole (PPO) as a fluor. This LAB-based organic liquid scintillator is referred to as the ``unloaded'' liquid scintillator.

The main physics goal of the scintillator phase is to measure low energy solar neutrinos: the CNO, pep and low energy $^8$B neutrinos. The pep neutrinos are mono-energetic, with $E_\nu$=1.442 MeV and their flux is well predicted by the Standard Solar Model\cite{davini2016cno}. A measurement of the pep neutrinos will give more information of the matter effects in neutrino oscillations. 

The solar metallicity is the abundance of elements heavier than $^4$He (called ``metal'' elements in the context of astronomy). It is poorly constrained and the predictions from different solar models disagree with each other. A measurement of the CNO neutrinos can give the abundance of $^{12}$C, $^{13}$N and $^{15}$O and can thus resolve the metallicity problem\cite{cerdeno2018cno}.

Three kinds of antineutrinos will also be measured: reactor antineutrinos as mentioned before; geoneutrinos from natural radioactivity in the Earth; and the supernova neutrinos. SNO+ is planned to join the SuperNova Early Warning System (SNEWS), which is an international
network of experiments with abilities of providing an early warning of a galactic supernova\cite{snop_nim}.

\subsubsection{Tellurium Phase}
In this final phase, 0.5\% natural Tellurium (Te) by mass (with 1.3 kilo-tonnes of $^{130}$Te) will be loaded into the scintillator, which is referred to as the ``Te-loaded'' scintillator. Higher loading concentrations would be possible for a further loading plan\cite{Paton:2019kgy}. The main purpose of this phase is to search for the $0\nu\beta\beta$ signals in $^{130}$Te.

\section{Detection Medium}
In the SNO+ detector, charged particles are expected to interact with the detection medium and create Cherenkov lights and scintillation lights. 

\subsection{Cherenkov Radiation}
For any charged particle travelling in a transparent medium at an ultrarelativistic speed (a speed greater than the local phase speed of light in the medium), an electromagnetic radiation, called Cherenkov radiation, can be emitted from the coherent response of the medium under the action of the field of the moving particle\cite{jackson2007classical,landau2013electrodynamics}.

Suppose a charged particle moves in a transparent, isotropic and non-magnetic medium and creates an electromagnetic wave. The electromagnetic wave propagates with a wave number $k=n\cdot\omega/c$, where $c$ is the speed of light in vacuum, $n$ is the real-valued refractive index and $\omega$ is the frequency. If the particle travels uniformly along x axis with a velocity of $v$, the x-component of the wave vector is $k_x=\omega/v$. For a freely propagating wave, $k>k_x$, therefore $v>v_p=c/n(\omega)$, where $v_p$ is the phase velocity in the medium. Under this condition that the speed of the charged particle is greater than the $v_p$, the Cherenkov radiation is emitted with a frequency of $\omega$\cite{landau2013electrodynamics}.   

The Cherenkov angle, $\theta_c$ is the angle between the direction of the particle and the direction of Cherenkov emission and it is well-defined by $\cos\theta_c(\omega) = \frac{c}{n(\omega)v}$. The radiation is distributed over a surface of a cone with the half-opening angle $\theta_c$. 

Consider the condition $v>v_p=c/n(\omega)$, for the case of $e^-$ travelling in a water detector, if neglecting the dependency on $\omega$, $n_{water}\simeq 1.33$ \cite{pdg2018}, then $\theta_c\simeq 41.25^\circ$ and $v_p\simeq 2.254\times10^8~m/s$, which corresponds to a kinetic energy $E_k=(\gamma-1)mc^2=0.264~MeV$ for $e^-$, where $\gamma=1/\sqrt{1-v_p^2/c^2}$. This is the lowest kinetic energy to create Cherenkov radiation, which is referred to the Cherenkov threshold ($E_{thresh}$). In the case that the LAB-PPO liquid scintillator is the medium, $n\simeq 1.50$\cite{tseung2011ellipsometric}, $\theta_c\simeq 48.19^\circ$ and for $e^-$, $E_{thresh}\simeq 0.175~MeV$.   

For a particle with a charge of $ze$, the number of photons produced by Cherenkov radiation per unit path length and per unit frequency of the photons is given by\cite{leo2012techniques}:
\[
\frac{d^2N}{d\omega dx}=\frac{\alpha^2 (ze)^2}{c}\sin^2\theta_c=\frac{z^2\alpha}{c}(1-\frac{1}{\beta^2 n^2(\omega)}),
\]
where $\alpha$ is the fine structure constant.

Translate the frequency into the wavelength ($\lambda=2\pi\omega$) and integrate over the wavelength, we have the number of photons and $x$ is along the particle track\cite{leo2012techniques}:
\[
\frac{dN}{dx}=2\pi (ze)^2\alpha\sin\theta_c\int_{\lambda_1}^{\lambda_2}\frac{d\lambda}{\lambda^2},
\]

For optical photons with wavelengths ranging from $350$ to $550~nm$ (typical PMT detection sensitive range), the above formula can be calculate into\cite{leo2012techniques}:
\[
\frac{dN}{dx}=476(ze)^2\sin^2\theta_c~photons/cm.
\]

For the Cherenkov radiation caused by $e^-$ in a water detector, $dN/dx \simeq 207~photons/cm$; while in the LAB-PPO case, $dN/dx \simeq 264~photons/cm$. In a realistic measurement, the detection efficiency and the coverage of photon sensors are also required to be taken into account.

\subsection{Scintillation from Organic Scintillator}

Besides the Cherenkov photons described in the previous section, the majority lights emitted from organic scintillator are scintillation photons.

The organic liquid scintillator can convert the kinetic energy of charged particles into scintillation photons with wavelengths in the sensitive detection region of PMTs. They are aromatic hydrocarbon compounds with benzene-ring structures. When ionizing radiation happens in the scintillator, the free valence electrons of the molecules are excited and transit to occupy the $\pi$-molecular orbitals with the benzene rings. These highly delocalized electrons are called $\pi$-electrons, which can occupy a series of energy levels. A Jablonski diagram, invented by Polish physicist Aleksander Jab\l o\'{n}ski, is generally used to describe molecular absorbance and emission of light. In Fig.~\ref{jablonski}, the Jablonski diagram illustrates the $\pi$-electronic energy levels of an organic scintillator molecule\cite{knoll2010radiation,leo2012techniques}. 
\begin{figure}[!htb]
	\centering
	\includegraphics[width=10cm]{jablonski.png}
	\caption{A Jablonski diagram for the organic scintillator, modified from \cite{birks1965theory, knoll2010radiation}.}
	\label{jablonski}
\end{figure}

In the diagram, $S_{0,1,2,3,...}$ are the energy levels of the spin-0 singlet states, where $S_0$ is the ground state and $S^*=S_{1,2,3,...}$ are the excited singlet states. Above the ground state $S_0$, there are also a set of spin-1 triplet states $T_{1,2,3,...}$, where $T_1$ is the lowest triplet state. These electron energy levels are labeled with thick black lines. The energy spacing between these levels are $\mathcal{O}(eV)$. In each levels, there are also fine structure levels which corresponds to excited vibration modes of the molecule (labeled with gray lines and can be marked as $S_{10}, S_{11}, ..., S_{20},S_{21}, ...$). The energy spacing between these fine levels are $\mathcal{O}(0.15~eV)$\cite{leo2012techniques, knoll2010radiation}.

The ionization radiation transfers the energy to the molecules and excites the electron levels as well as the vibrational levels, labelled as the absorption lines (in green). The decays between the excited singlet states (not to the ground state) are almost immediate ($\leq 10~ps$) without the emission of light. This process is called internal degradation. The decays from the excited singlet state $S_1$ (as well as the vibrational states $S_{10},S_{11},S_{12},...$) to the ground state (as well as the vibration states $S_{01}, S_{02}, ...$) happen promptly ($\mathcal{O}(ns)$) and emit lights (labelled as red lines). This process is called fluorescence which contributes the prompt component of the emission of scintillation light. The probability of $S_1$ decays into the vibrational states $S_1 \to S_{01},S_{02},...$ among the ground state is more than $S_1\to S_0$. Since the absorbed energy of $S_0 \to S_1$ is larger than the emitted energy of $S_1 \to S_{01},S_{02},...$, the scintillators have very little self-absorption of the fluorescence and are transparent to their own radiation. The effect of Stokes shift, which refers to the overlap between the optical absorption and emission spectra, is small for the organic scintillator\cite{leo2012techniques,knoll2010radiation}. 

The transitions between the singlet and triplet states are highly forbidden due to the
electron spin-flip is involved\cite{von2015measurement,sorensen2016temperature}. There also exists a relatively rare process called intersystem crossing (ISC), which converts excited singlet states into triplet states. Besides this, 75\% of triplet states can be produced by ionization-recombination\cite{von2015measurement,dunger2018topological}.

For the de-excitation, the similar processes of internal degradation occur among $T_{2,3, ...} \to T_1$. $T_1$ is a relatively stable state and the lifetime of the molecule in the triplet state is in $\mathcal{O}(10^{-4}~-~10~s)$\cite{mcquarrie1997physical}. $T_1\to S_0$ is highly forbidden. However, the $T_1$ state can go through an indirect decay process by interacting with another excited $T_1$ molecule and forms an excited singlet state:
\[
T_1+T_1\to S^*+S_0+phonons
\]
The $S^*$ will de-excite and emit delayed scintillation light. The process for emitting this delayed scintillation light is called delayed fluorescence or phosphorescence\cite{leo2012techniques}. This process contributes to the delayed component of scintillation light.

For a typical scintillator detector, the time scale of detector response is $\mathcal{O}(1-100~ns)$. In this time region, the emission of the scintillation light contains the primary fluorescence from the de-excitation of the singlet states (prompt component) and the delayed fluorescence from the de-excitation of the indirect triplet states (delayed component)\cite{dunger2018topological}. The time profile of the scintillation light emission is a mixture of prompt and delayed components. 

Different charged particles can cause different ionization densities when they deposit energies to the scintillator molecules. The ionization density affects the relative population of the excited singlet and triplet states. Compared to an $e^-$, an $\alpha$ particle can cause a high ionization density, which produces higher ratio of triplet states. Therefore, the time profile for the $\alpha$ particle has more delayed component or longer tails than the $e^-$. This enables the organic scintillator to distinguish $\alpha$ with $e^-$ or other lighter charged particles\cite{dunger2018topological, scintillatorPaper}. 

An empirical formula, called follows Birk's law\cite{birks1951scintillations, birks1965theory}, describes the photon yield along unit distance by the incident particle:
\[
\frac{dY}{dx}=A\frac{dE/dx}{1+k_B\cdot dE/dx},
\]
where $A$ is a normalization constant, $k_B$ is the Birks' constant of the scintillator, which in practice is obtained by fitting the formula to the measured data.

\subsection{Liquid Scintillator for SNO+}
Organic scintillators can release a large amount of photons with wavelengths in the sensitive regions of the PMTs and have abilities for particle identification. In addition, since organic liquids are non-polar media, it is hard for ionic impurities to dissolve in. This leads to the lower contamination levels of uranium (U), thorium (Th) and potassium (K) in the organic liquid scintillators. Among the organic scintillators, aromatic organic liquid scintillators have high electron densities which cam provide sufficient target for particle interactions\cite{PerkinElmer}. Due to these advantages, aromatic organic liquid scintillators have been extensively developed as detection media for large particle detectors, especially for neutrino experiments, such as KamLAND, Borexino, Day Bay and JUNO\cite{scintillatorPaper}.

SNO+ has developed such kind of liquid scintillators that are compatible with the detector components, especially with the acrylic materials, such as the AV sphere. Two kinds of SNO+ liquid scintillators are discussed in the following sub-sections: the unloaded liquid scintillator for the scintillator phase and the Tellurium-loaded liquid scintillator (TeLS) for the tellurium phase.

\subsubsection{Unloaded Liquid Scintillator}
SNO+ adopts a scintillator cocktail contains two primary components: linear alkylbenzene (LAB) as solvent and 2,5-diphenyloxazole (PPO) as solute. For per liter LAB, 2 grams of PPO is dissolved. Fig.~\ref{labppo-molecule} shows the chemical structural formulae of LAB and PPO\cite{scintillatorPaper}.
\begin{figure}[!htb]
	\centering
	\includegraphics[width=8cm]{lab-ppo-molecule.png}
	\caption{Structural formulae of LAB (left) and PPO (right).}
	\label{labppo-molecule}
\end{figure}

LAB is a family of alkylated aromatic organic compounds with a phenyl group attached to a long carbon chain varying from 9 to 14 carbons\cite{wiki_LAB, scintillatorPaper}. It has been used as a biodegradable detergent since the 1960s and it is proved to be relatively non-toxic and very low risks for the environment or human health\cite{wiki_LAB}.





LAB is an effective energy absorber.

 good optical transparency 



 flash point at 140 $\circ C$

%Kosswig, Kurt (2005). "Surfactants". Ullmann's Encyclopedia of Industrial Chemistry. Wiley-VCH. doi:10.1002/14356007.a25_747. ISBN 3527306730.



The LAB is provided by CEPSA Qu\'{i}mica B\'ecancour Inc.

The advantages of LAB are:
\begin{itemize}
	\item[$\bullet$] It has very low levels of natural radioactive contaminants such as U, Th and K.
	The target background levels for the SNO+ LAB are expected to be $\mathcal O(10^{-17})$ gram of $^{238}$U in per gram LAB (g$^{238}$U/gLAB), which is equal to be thousands of events per year;
	The $^{232}$Th and $^{40}$K levels are $\mathcal O(10^{-17})$ g/gLAB, which is equal to be hundreds of events per year\cite{markchen_bkg}.
	
	\item[$\bullet$] High light yield and attenuation length.
	\item[$\bullet$] It has fast timing response different timing spectrum for $\alpha$ and $\beta$ events, which enables an $\alpha/\beta$ discrimination. 
	\item[$\bullet$] High flash point and low toxicity for lab safety.
	\item[$\bullet$] appropriate density for mechanical stability
	\item[$\bullet$] Good stability and chemically compatible with detector materials, mainly the AV.
	\item[$\bullet$] Low cost.
\end{itemize}

As a wavelength shifter, PPO is usually added and dissolved into the LAB \cite{wunderly1990new}. This wavelength shifter is used as a fluor

energies are transferred from LAB to PPO via 


 non-radiative F{\"o}rster resonant energy transfer



it can shift the wavelengths of the scintillation photons to a range of 300-550 nm, which is the 

sensitive range of the PMT detection.  A 2 g/L PPO concentration in LAB is optimized by SNO+\cite{whitepaper}. The absolute light yield of the LAB-PPO liquid scintillator has been well-measured from large particle physics experiments \cite{xing2015preliminary}, borexino] as well as bench-top measurements \cite{xing2015preliminary}, novikov, tanner]. The absolute light yield of LAB+2g/L PPO liquid scintillator determined by SNO+ is 11900 photons/MeV [cite lightyield].






Fig.~\ref{absLength} shows absorption lengths.

\begin{figure}[!htb]
	\centering
	\includegraphics[width=10cm]{absLength.png}
	\caption{Absorption length of SNO+ optical components. ;the internal (solid blue line) and external water (dashed blue line) absorption curves are based on the measurements of the laserball scans in July 2018 during the SNO+ water phase. The horizontal lines are due to absence of the measurements and they are conservative assumptions.}
	\label{absLength}
\end{figure}







non-radiative transfer from LAB (solvent) to PPO (fluor)


the non-radiative transfer efficiency for 2g/L PPO in LAB is measured as $78.2\pm 1.5~\%$

the transfer efficiencies


This transfer efficiency ($\mathcal{\epsilon}_{transfer}$) increases as the PPO concentration increasing, as shown in the Table. Above the 2 g/L concentration, the light yield reaches a plateau
\cite{scintillatorPaper}.

\ref{transfer_efficiency}\cite{scintillatorPaper}. 
\begin{table}[ht]
\centering
	\begin{tabular}{cc}
		\toprule
concentration (g/L) & $\mathcal{\epsilon}_{transfer}$(\%)	\\
\midrule
4   & $86.0\pm 0.8$\\
2  & $78.2\pm 1.5$\\
1  & $67.7\pm 2.3$\\
0.5 & $59.3\pm 3.2$\\
0.25 & $48.7\pm 5.0$\\
		\bottomrule
	\end{tabular}
	\label{transfer_efficiency}
\end{table}










\subsubsection{Tellurium-loaded Liquid Scintillator}

To load the Te into the liquid scintillator, a compound is made by 
condensation reactions between telluric acid (TeA) and 1,2-butanediol (BD), with N,N-dimethyldodecylamine (DDA) being used as a stabilization agent.

A tertiary amine (N,N-Dimethyldodecylamine, DDA) was added during the reaction to stabilize TeBD complexes and avoid any phase separation. 


Tellurium-loaded 65\% of the pure, unloaded scintillator



water-based wavelength shifter


timing profile, the intensity of scintillation light as a function of time

the prompt fluorescence intensity at a time $t$ excitation be $I=I_0e^{-\frac{t}{\tau}}$



singlet and triplet states 
ionization density 
depend
$\alpha$-particle
high ionization density 
quenching, 




2 g/L PPO gives an absolute light yield of 11900 photons/MeV.


for the partial-fill phase, 0.5 g/L PPO gives Measurements in 0.5 g/L showed a light yield of 52\% of 2 g/L,  
6190 photons/MeV\cite{tanner0p5,joshW1}.





\section{Optics}

Optical parameters

Winston cone



timing


attenuation

scattering


laser pulse diffuser, it can run with different wavelengths: 337, 365, 385, 420, 450 and 500 nm.
The laserball 

The acrylic of the AV is UV-transparent

\section{Electronics}
In this section, the SNO+ electronics system is introduced. The system includes trigger and readout systems. As mentioned in \ref{overview}, the PMTs as photon sensors are the basic detection elements for the SNO+ detector. The signals from the PMTs are sent to the SNO+ electronics system, which records the PMT time and charge information and then transfers the digitized data to offsite computing systems for data analysis. These steps are detailed in the following.

The photons created from particle interactions in the detector propagate to the PMT sphere and may hit a certain PMT and strike on its photo-cathode, which is a thin caesium bialkali film coated on the inner surface of PMT glass. The photocathode then produces a photo-electron (p.e.) through photoelectric effect. The photocathode is set at ground voltage while the anode is at a high voltage ranging from $+1700$ to $+2100~V$ \cite{boger2000sudbury,dunger2018topological}. This forms electric fields inside the PMT. The p.e. is accelerated and focused by the electric field in the PMT and goes through the volume which is under vacuum until it reaches the region of a series of secondary emission electrodes, called dynodes. The nine dynode strings are constructed in a Venetian blind configuration in R1408 PMT \cite{boger2000sudbury,leo2012techniques}. When the p.e. transfers its energy to the materials in dynodes, a number of secondary electrons escape and form a measurable current which is collected by the a custom-made operating circuit (PMT base) at the anode\cite{hamamatsu2018photomultiplier}.

The anode pulse produced from the PMT travels along 35 m-long RG59/U type coaxial cable (with a resistance of 75 $\Omega$) to the front-end electronics which are set up on the deck above the detector. The coaxial cable also carries the high-voltage\cite{boger2000sudbury}. 

To tackle with more than 9000 PMTs in the SNO+ detector, the coaxial cables connected to each PMTs are grouped into bundles. Each bundle with 8 PMTs is connected to a Paddle Card. Four Paddle Cards are linked to a PMT Interface Card (PMTIC). The PMTIC supplies high voltages and receives the signals of up to 32 channels. Each channel for one PMT. 32 channels in PMTIC are plugged to a Front End Card (FEC) that processes, digitizes and stores PMT signals. Each FEC has 4 Daughter Boards (DB) and each DB handles 8 channels. 16 PMTICs and FECs are inserted in one electronic crate and thus each crate processes 512 PMTs. 19 crates tackle 9728 PMT channels in total, of which 32 channels are reserved for calibration inputs and labelled as FEC Diagnose (FECD) channels. The triggered PMTs can be labelled by the logical channel number (lcn) using the map of the PMT to the crates and cards\cite{snop_nim,stringer2019sensitivity}:
\begin{equation}
lcn = 512 \times crate + 32 \times FEC + channel
\end{equation}
 %9605 are actually used
 
To properly process and record information from PMTs, each DB tackling with 8 channels has three types of custom-made integrated circuits to implement three functions\cite{boger2000sudbury,bonventre2014neutron,snop_nim}:
\begin{itemize} 
\item[$\bullet$] Two four-channel discriminator (SNOD) chips set thresholds for PMT signals.
\item[$\bullet$] Eight single-channel CMOS chips record the time when the PMT pulse cross the discriminator thresholds.
\item[$\bullet$] Two eight-channel charge integrator (SNOINT) chips integrate the PMT pulses over time interval to obtain PMT charge information. There are three different time windows for the integral, resulting in three types of charge values: QHS (Charge High gain Short integration time), by integrating the pulse over 60 ns; QHL (Charge High gain Long integration time), by integrating the pulse over 400 ns; and the QLX (Charge Low gain),by integrating the the low gain PMTs over either 60 or 400 ns.
\end{itemize}

First, the discriminator checks the input pulses from a PMT channel with the predefined threshold voltages. If the PMT pulse crosses the channel threshold, a discrete square trigger pulse is generated. The trigger pulse has different time widths for different trigger types, including a 93 ns long square pulse (N100) and a 48 ns square pulse (N20)\cite{joshTrigger}. These time widths are optimized for the timing of the photon propagation in the detector and later to determine the number of hit PMTs in a specific time window. Beside these two, there are also two triggers which copies the analog pulses of high and low gain of each hit PMTs, known as ESUMHi(gh) and ESUMLo(w). These triggers mainly measure the stability of the detector, in case that there would be large signals in the PMTs while the number of the hit PMTs are small due to one or a few PMTs being exposed to high amount of light or experiencing break downs\cite{operator}. At the same time, CMOS chips produces a time-to-amplitude (TAC) voltage and the TAC starts to ramp. The SNOINT starts to integrate all pulses from the PMT channel from 10 ns before the threshold crossing time to 390 ns or 50 ns to get QHL, QHS or QLX values mentioned before\cite{boger2000sudbury,stringer2019sensitivity}. 

The trigger pulses from all the channels in one crate with 16 FECs are sent to a Crate Trigger Card (CTC) and are summed up. For 19 crates, 19 CTCs are connected to 7 Master Trigger Cards/Analogue (MTC/A+) (``+'' refers to SNO+'s upgrading from the SNO), which are installed in an individual rack. It takes about 110 ns for the signals to travel to the MTC/A+. The MTC/A+ sums the trigger pulses from all crates and checks whether the sum is above the trigger threshold. Then the summed trigger pulse is sent to the digital master trigger system (MTC/D) which checks 7 MTC/A+ triggers. As mentioned in the previous paragraph, the summed triggers of N100 and N20 indicate the number of hit PMTs within the pulse time widths. In the MTC/A+ discriminators, there are three different threshold values of the number of the hit PMTs for the summed trigger pulses: Lo(w), Med(ium) and Hi(gh). During the water phase, N100Lo, N100Med, N100Hi are set based on the N100 pulses, where each has a different threshold value set. from lower value to higher value (for example, a setting of N100Lo=11, N100Med=16, N100Hi=21 was used in the water phase simulations). A lower threshold can increase the data-taking rates by including more signals as well as background noise. A high event rate can also make the detector unstable. On the other hand, a higher threshold removes more noises with a sacrifice of signals. The MTC/D checks the trigger type with the pre-defined trigger mask (a set of multiple trigger types with different trigger thresholds can be set), if it is satisfied, a Global Trigger (GT) is issued\cite{snop_nim,stringer2019sensitivity,rattime}. 

Once the GT is issued, an event is triggered and the MTCD sends the GT to the FEC which ramps the TAC within the next 20 ns (determined by a 50 MHz clock mentioned later). This takes another 110 ns. Thus there is a total delay of 220 ns after the FEC ramping the TAC. When the TAC ramps, a time gate of 400 ns is set for waiting the GT. If there is a GT in this 400 ns window, the TAC is stopped and resultant voltage is digitized and recorded as the PMT hit time. If there is no GT, the TAC resets and the PMT hit is not recorded. Meanwhile, for the square trigger pulses from each PMT channels, if they are within the 400 ns gate and before the GT arrives, they will be recorded as the hits of an event. The recorded number of the hit PMTs for a given event is referred to as NHits. The charge information of the event is also recorded\cite{stringer2019sensitivity,rattime}.

A 10 MHz and a 50 MHz clocks are used to record the time of the triggered event. The universal time of the triggered event is calculated as the time elapsed from a predefined $T_{zero}, $the midnight of January 1, 2010 (GMT) to the moment when the event happens.A 10 MHz clock used for counting the absolute time started at $T_{zero}$. It has a 53 bit register and can run for 28.5 years. Its accuracy is maintained by a GPS system. The 50 MHz clock gives more accurate timing. It limits the best time resolution of the GT to 20 ns. This clock has a 43 bit register and rolls over every 2.04 days. The relative time between the events can be used for analyzing specific physics processes, such as radioactive decays\cite{rattime,stringer2019sensitivity}. 

The recorded hit information of the triggered event, including the values of the TAC, QHS, QHL, QLX (i.e., the time and charge information of hit PMTs of the event) and the trigger settings are sent to a Crate Controller Card (XL3) in each crate. These cards were installed for SNO+ to handle higher data transfer rates compared to SNO, with a max rate of 14 MB/s, which is equivalent to approximately 2 million hits per second\cite{bonventre2014neutron}. They read out the recorded data and wrap them as ethernet packets and send to the Data Acquisition System (DAQ) and Event Builder system\cite{walker2016study}. The Event Builder system writes these information into event records based on their GT identification number (GTID) and saves them on storage disk\cite{snop_nim}. These raw data are written in the ZEBRA data bank (ZDAB) format and then are further processed into ROOT format by high-performance computing clusters.

As a summary, the SNO+ electronic system can measure signals with a nanosecond-level timing resolution and a single-photon level charge resolution. It can handle an event rate of several kHz and even much higher rates for cases such as the burst events from a galactic supernova\cite{snop_nim}.

\section{Calibration}
Two kinds of calibration sources are used by SNO+: optical sources and radioactive sources. 
The optical sources are used to calibrate the PMT response and to measure the optical properties of the 
detector media 

Radioactive source: energy scale, resolution, systematic
uncertainties

reconstruction performances and uncertainties.
particle identifications

Calibration sources with known physics parameters: help to understand the detector response to the events and to make accurate measurements

the SNO+ Source Manipulator System (SMS) is inherited from the SNO.

A Umbilical Retrieval Mechanism (URM) is used to send the source down to the inner vessel.

The sources are connected to the umbilical.

An umbilical encloses electrical cables, optical fibres and gas lines connected to the source.

A Universal Interface (UI) connecting the URM and the detector, 
Therefore, sealed environment, which 
ensures radon gas not leaking into the detector when deploying the source.

The source manipulator system (see Section 6.3) allows the insertion of various types of calibration
devices inside the acrylic vessel. 

\subsection{The $^{16}$N Calibration Source}


 is one of the radioactive sources

This source is inherit from SNO experiment\cite{dragowsky1999sudbury,dragowsky200216n,hamer2001energy}, 

A deuterium-tritium (DT) generator in SNOLAB can produce neutrons through: $D+T\to n+^{4}$He, 
flow $CO_2$ gas stream through pipe lines
the $^{16}$N isotopes are created by the process: $n+^{16}$O$\to^{16}$N$+p$,


The $^{16}$N isotope mainly decays through $\beta$-decay process: $^{16}$N$\to ^{16}$O$+e^-+\bar{\nu}_e$.
It has a 66.2\% chance to emit an electron with an end-point energy of 4.29 MeV and 22.8\% chance to 
10.42 MeV

\cite{nndc}

a simplified decay scheme is shown in Fig.~\ref{n16decay}.

6.13 MeV $\gamma$ rays.
\begin{figure}[!htb]
	\centering
	\includegraphics[width=6cm]{n16_decay.png}
	\caption{$^{16}$N main decay scheme, modified from \cite{dragowsky200216n}.}
	\label{n16decay}
\end{figure}


Fig.~\ref{n16pic} shows the geometry of the $^{16}$N source chamber. The chamber is a stainless steel cylinder mainly containing a small PMT and a gas decay chamber. The chamber was designed to confine the electrons from $^{16}$N decay within the chamber and let them be detected by the PMT inside; 


to ensure a high fraction of the $\gamma$-rays 


lighthouse where the liquid in the scintillator volume (for example, pure water in the water phase) is free to enter.


tagged by a small PMT inside

A polyethelene bumper cone is at the bottom of the source.

gas capillary tube 

\cite{dragowsky1999sudbury}.


\begin{figure}[!htb]
	\centering
	\includegraphics[width=10cm]{n16geom.png}
	\caption{$^{16}$N calibration source geometry. Left: a detailed diagram of $^{16}$N source geometry, modified from \cite{maclellan2009energy,matt_deployedsource}; middle: source geometry implemented in RAT, modified from \cite{n16geom_zach}; right: a picture of the $^{16}N$ source, taken from \cite{n16pic}.}
	\label{n16pic}
\end{figure}

The $^{16}$N source
$^{3}$H$(p,\gamma)^{4}$He reaction.




\subsection{The Americium Beryllium (AmBe) Calibration Source}

Optical calibration  {\emph {in-situ}} 
\begin{itemize}  
	\item[$\bullet$] Timing module for the Embedded LED Light Injection Entity (TELLIE)
	
	light-emitting diode (LED)
	
	
	time calibration, time response
	calibrates the gain and charge response of the PMTs, which is important to estimate the event energy.
	
	a precision of $\mathcal{O} (1~ns)$
	
	Blinky fibre optics nailed to the PSUP to calibrate stuff.
	
	AMELLIE
	optical attenuation in the detector over time
	
	SMELLIE scattering of the detector media
	
	\item[$\bullet$]  
	
	
	\item[$\bullet$] 
\end{itemize}


\subsection{The Laserball}

optical source, named as laserball, is a light diffusing sphere, which nitrogen dye laser glass bubbles.



fast pulsing LED or lasers
measures optical properties, such as scattering, attenuation of the detector materials, the response of PMTs, 
angular and wavelength dependent
wavelength dependent absorption and the optical degradation

In the scintillator and tellurium phases, three additional $\gamma$ sources: $^{48}$Sc, $^{137}$Cs and $^{57}$Co will be implemented.

$^{46}$Sc source

\section{Detector Backgrounds}

\subsection{Physics Backgrounds}

Radioactive Isotopes

assays


in-situ
ex-situ



the instrumental backgrounds can be removed by the data-cleaning approaches.



\subsection{Instrumental Backgrounds}




\section{Monte Carlo Simulation and the RAT Software}
The SNO+ collaboration has developed a software framework, called the Reactor Analysis
Tool (RAT), which integrates a Monte Carlo simulation of the SNO+ detector and event-based analysis tools (for online and offline event analysis) since the beginning of the program. This software was originally developed by Stan Seibert for Braidwood Collaboration for a generic KamLAND like detector. It is also used by the other astroparticle physics experiemnts, such as DEAP/CLEAN, CLEAR and potentially for Darkside-50\cite{rat}.

Geant4 Toolkit and
incorporates ROOT libraries for data handling and analysis.

The RAT Monte Carlo was originally developed for the Braidwood Collaboration
utilising software developed for a generic KamLAND like detector, called Generic Liquid Scintillator GEANT4 simulation (GLG4). It was branched and
developed into the specialist SNO+ version, starting in 2006/2007.

GLG4sim


Braidwood

combines both Monte Carlo simulation of the Braidwood detector with event-based analysis tasks, like reconstruction. The primary goals are:

Make it easy to analyze Monte Carlo-generated events as well as data from disk using the same software with only a few command changes. Even in the proposal R\&D phase, where there is no real data, this is still useful for dumping Monte Carlo events to disk to be analyzed by another job. When there is real data, being able to do the analysis with the same code path as was used on Monte Carlo is very reassuring.
Allow for a modular, user-controlled analysis of events. This includes allowing the user to selected which analysis tasks to perform (different fitters, pruning unneeded data from the event data structure, etc.). It should also be relatively straightforward for users to introduce their own code into the analysis process.
Separate analysis into small tasks which can be developed asynchronously by different people, yet integrated with minimal (or perhaps zero) pain.
Integrate into existing GEANT4 and GLG4sim efforts with a minimum of code duplication. As much as possible, RAT should be designed incorporate upgrades of these packages just by relinking. No cut and paste of code (mainly a question with GLG4sim).
Design

the detailed processes of data acquisition and trigger systems are \cite{whitepaper}.

The SNO+ version of RAT is being developed by the whole collaboration and evolves with the experiment progress.
Besides more functions are added into the RAT, it is also tuned and optimized with the updated parameters from more precise descriptions of the physics processes or the detector responses to the calibration sources. Therefore, different versions of RAT may give different results. For the work in this thesis, multiple RAT versions are used, mainly the versions for the water phase and partial-fill phase. In this case, I will specify the RAT version when I discuss a certain analysis.
\cite{ratManual}